
\begin{appendices}

\chapter{FEM code}
\label{AppA}
\section{main.py}


\begin{verbatim}
## @package main
#  This file is used to run the script and call al functions

import numpy
from FEM  import *
from IOlib import *
from MeshDat import *
from output import *


E=70e9
nu=0.33

parameter=Parameter(E, nu)

mesh = read_from_txt('Mesh/MultiHole.msh')

Force = -1e10

dis_force = Distribute_Force(mesh,Force)

K = getK(mesh,parameter)
F = getF(mesh,dis_force)

U = solveSys(mesh,F,K)
sig=get_FEM_stresses(mesh,U,parameter)

plot_solution( mesh,'output', 'output',U,sig)

\end{verbatim}


\section{MeshDat.py}
\begin{verbatim}
# -*- coding: utf-8 -*-
## @package MeshDat
#  This file contains all objects

"""
Created on Wed Apr 06 17:44:14 2016

@author: Mark
"""

import numpy 

## Add nodes to a list before creating the mesh.
#  @param nlist  List of nodes
#  @param ID     Node ID
#  @param coord  Node coordinate
def addNode( nlist, ID, coord ):
    index = len(nlist)
    NewNode = Node(ID, coord, index)
    return nlist.append( NewNode )

## Add elements to a list before creating the mesh.
#  @param elist  List of elements
#  @param ID     Element ID
#  @param Enodes Element nodes
def addElement( elist, ID, Enodes ):
    NewElement = Element(ID, StandardTriangle(), Enodes)
    return elist.append( NewElement )

       
#==============================================================================#
## node object
class Node:
    
    ## Constructor
    #  @param ID    Node ID
    #  @param coord Node coordinate
    #  @param index Node index
    def __init__ ( self, ID, coord, index ):
        self.__ID = ID
        self.__index = index
        self.set_coordinate( coord )
        self.set_constraint( numpy.array([0 , 0]) )

    ## Get the Node ID
    def get_ID ( self ):
        return self.__ID
    
    ## Get the Node index    
    def get_index( self ):
        return self.__index
    
    ## Set the coordinate
    #  @param coord Node coordinate   
    def set_coordinate( self, coord ):
        assert isinstance( coord, numpy.ndarray )
        assert coord.ndim==1
        assert coord.dtype==float
        self.__coord = coord

    ## Get the Node coordinate
    def get_coordinate( self ):
        return self.__coord
        
    ## String function
    def __str__ ( self ):
        s  = 'Node ID           : %d\n' % self.get_ID()
        s += 'Node coordinates  : %s\n' % numpy.array_str(self.get_coordinate())
        s += 'Node index        : %d\n' % self.get_index()
        return s
    
    ## Set Constraint
    # @param cons Set constraint for node
    def set_constraint( self, cons ):
        assert isinstance( cons, numpy.ndarray )
        assert cons.ndim==1
        assert cons.dtype==int
        self.__cons = cons
        
    ## Get Constraint
    def get_constraint( self ):
        return self.__cons

    ## Set Force Constraint
    # @param consF set Force constraints
    def set_forcecons( self, consF ):
        assert isinstance( consF, numpy.ndarray )
        assert consF.ndim==1
        assert consF.dtype==float
        self.__conF = consF

    ## Get ForceConstraint
    def get_forcecons( self ):
        return self.__conF

#==============================================================================
## element object
class Element:

    ## Constructor
    #  @param ID     Element ID
    #  @param parent Standard/parent element
    #  @param nodes  List of finite element Nodes
    def __init__ ( self, ID, parent, nodes ):
        self.__ID     = ID
        self.__nodes  = nodes
        self.__parent = parent
        self.__nnodes = len(parent)
        assert len(parent)==len(nodes)

    ## Get element ID
    def get_ID(self):
        return self.__ID
    
    ## Get nodes
    def get_nodes( self ):
        return self.__nodes
        
    ## Get node IDs from element
    def get_node_IDs( self ):
        IDlist = []
        for node in self.__nodes:
            IDlist.append(node.get_ID())
        return IDlist     
            
    ## Get the number of nodes
    def get_nr_of_nodes ( self ):
        return self.__nnodes
        
    ## Iterator function
    def __iter__( self ):
        return iter(self.__nodes)

    ## Length function
    def __len__( self ):
        return self.get_nr_of_nodes()

    ## String function
    def __str__( self ):
        s  = 'Element %d with nodes: ' % self.__ID
        s += ', '.join(str(node) for node in self.__nodes)
        return s
        
    ## Get item function
    def __getitem__ ( self, index ):
        return self.__nodes[index]
        
    ## Get the jacobian
    #  @param  xi Local coordinate vector
    #  @return    Jacobian matrix
    def __get_jacobian ( self, xi ):
        coords          = self.get_coordinates()
        std_shapes_grad = self.__parent.get_shapes_gradient(xi)
        return coords.T.dot( std_shapes_grad )
        
    ## Get the integration scheme
    #  @param  name The type of integration scheme (e.g. 'gauss')
    #  @param  npts The number of integration points
    #  @return      Matrix of integration point coordinates
    #  @return      Vector of integration point weights
    def get_integration_scheme ( self, name, npts ):
        xis, ws = self.__parent.get_integration_scheme( name, npts )
        ws = numpy.array([w*numpy.linalg.det(self.__get_jacobian( xi )) for xi, 
			w in zip(xis,ws)])
        return xis, ws
        
    ## Get the shape functions gradient
    #  @param  xi Local coordinate vector
    #  @return    Matrix of shape function gradients
    def get_shapes_gradient ( self, xi ):
        J_inv = numpy.linalg.inv( self.__get_jacobian( xi ) )
        std_shapes_grad = self.__parent.get_shapes_gradient(xi)
        return std_shapes_grad.dot( J_inv )
        
    ## Get the matrix of nodal coordinates
    def get_coordinates( self ):
        return numpy.array([node.get_coordinate() for node in self])
        
    ## Get the global coordinate
    #  @param  xi Local coordinate vector
    #  @return    Global coordinate vector
    def get_coordinate ( self, xi ):
        coords     = self.get_coordinates()
        std_shapes = self.__parent.get_shapes( xi )
        return coords.T.dot( std_shapes )
        
#==============================================================================
        
## Finite element mesh
class Mesh:
    ## Constructor
    #  @param nodes list of finite element nodes
    #  @param elems list of finite elements
    #  @param LSnodes list of boundary nodes on left side of geometry
    #  @param RSnodes list of boundary nodes on right side of geometry
    def __init__ ( self, nodes, elems, LSnodes, RSnodes ):
       self.__nodes = nodes
       self.__elems = elems
       self.__LSnodes = LSnodes
       self.__RSnodes = RSnodes
              
    ## Get a node
    #  @param ID Node ID
    def get_node ( self, ID ):
        for node in self.__nodes:
            if node.get_ID()==ID:
                return node
        raise RuntimeError( 'Node ID %d not found' % ID )
        
    ## Iterator function
    def __iter__ ( self ):
        return iter(self.__elems)

    ## Length function
    def __len__ ( self ):
        return len(self.__elems)

    ## String function
    def __str__ ( self ):
        s  = 'Number of nodes   : %d\n' % self.get_nr_of_nodes()
        s += 'Number of elements: %d\n' % len(self)
        return s
        
    ## Get the list of nodes in the mesh
    def get_nodes( self ):
        return self.__nodes
        
    ## Get the list of elements in the mesh
    def get_elems( self ):
        return self.__elems

    ## Get the list of nodes on the left boundary of the mesh
    def get_LSnodes( self ):
        LSnodelist = []
        for i in self.__LSnodes:
            LSnodelist.append(self.get_node(i))
        return LSnodelist
        
    ## Get the list of nodes on the right boundary of the mesh
    def get_RSnodes( self ):
        RSnodelist = []
        for i in self.__RSnodes:
            RSnodelist.append(self.get_node(i))
        return RSnodelist

    ## Get the number of nodes
    def get_nr_of_nodes ( self ):
        return len(self.__nodes)

    ## Get the number of elements
    def get_nr_of_elements ( self ):
        return len(self.__elems)
        
    ## Get the number of boundary nodes
    def get_nr_of_LSnodes ( self ):
        return len(self.__LSnodes)
        
    ## Get the number of boundary nodes
    def get_nr_of_RSnodes ( self ):
        return len(self.__RSnodes)

    ## Get the number of constrains
    def get_nr_of_constraints( self ):
        total = 0
        for node in self.__nodes:
            total += node.get_constraint().sum();
        return total

    ## Get the number of constrained nodes
    def get_nr_of_nodes_with_constraints( self ):
        total = 0
        for node in self.__nodes:
            if node.get_constraint().sum()>0:
                total += 1
        return total
#==============================================================================
    
##  Linear triangle parent element with local coordinates (0,0), (1,0), (0,1)
class StandardTriangle:
    
    ## Dictionary of integration schemes
    #
    #  See e.g. 'http://www.cs.rpi.edu/~flaherje/pdf/fea6.pdf' for details
    __ischemes = {
                   ('gauss',1 ) : ( numpy.array([[1./3.,1./3.]]),
                                    numpy.array([.5]) )
                 }

    ## Number of nodes
    __nnodes = 3

    ## Length function
    def __len__ ( self ):
        return self.get_nr_of_nodes()

    ## Get the number of nodes
    def get_nr_of_nodes ( self ):
        return self.__nnodes

    ## Get the shape functions
    #  @param  xi Local coordinate vector
    #  @return    Vector of shape functions
    def get_shapes ( self, xi ):
        return numpy.array([1-xi[0]-xi[1],xi[0],xi[1]])

    ## Get the shape functions gradient
    #  @param  xi Local coordinate vector
    #  @return    Matrix of shape function gradients
    def get_shapes_gradient ( self, xi ):
        return numpy.array([[-1.,-1.],
                            [ 1., 0.],
                            [ 0., 1.]])

    ## Get the integration scheme
    #  @param  name The type of integration scheme (e.g. 'gauss')
    #  @param  npts The number of integration points
    #  @return      Matrix of integration point coordinates
    #  @return      Vector of integration point weights
    def get_integration_scheme ( self, name, npts ):
        xis, ws = self.__ischemes[ (name,npts) ]
        return xis, ws

    ## Get the element's internal connection scheme
    def get_connections_scheme (self):
        return numpy.array([[ 0 , 1 ],
                           [ 1 , 2 ],
                           [ 2 , 0 ]])

\end{verbatim}

\section{IOlib.py}
\begin{verbatim}
## @package IOlib
#  This file is used to read the mesh from a .msh file.
# -*- coding: utf-8 -*-

"""
Created on Tue Apr 05 17:53:24 2016

@author: Mark
"""

import numpy 
from MeshDat import *


## Mesh file reader
#
#  @param  fname Name of the mesh file
#  @return       Finite element mesh
def read_from_txt( fname ):
    try:

        #Open the file to read
        #fname = 'SingleHole_Circle.msh'
        fin = open( fname )
        
        # this skips the $Physical names section
        line     = fin.readline()
        line     = fin.readline()
        line     = fin.readline()
        line     = fin.readline()
        
        # Read the number of PhysicalEntities
        linelist = line.strip().split()
        assert linelist[0] =='$PhysicalNames'
        line     = fin.readline()
        nPhysical = int(line)
        for inode in range( nPhysical + 1 ):
            line     = fin.readline()
            
        linelist = line.strip().split()
        
        #Read the number of nodes
        line     = fin.readline()
        linelist = line.strip().split()
        assert linelist[0] =='$Nodes'
        line     = fin.readline()
        nnodes = int(line)
        
        nodes   = []
        nodeIDs = []
        for inode in range( nnodes ):
            linelist = fin.readline().strip().split()
            nodeID   = int(linelist[0])
            coord    = numpy.array(linelist[1:3],dtype=float)
            addNode(nodes, nodeID, coord)   
            nodeIDs.append( nodeID )
            
        line     = fin.readline()
        
        #Read the number of elements
        line     = fin.readline()
        linelist = line.strip().split()
        assert linelist[0] =='$Elements'
        line     = fin.readline()
        nelems   = int(line)
        
        elems = []  
        LSnodes = []
        RSnodes = []
        for ielem in range( nelems ):
            linelist    = fin.readline().strip().split()
            elemID    = int(linelist[0])    
            enodesIDs = numpy.array(linelist[5:],dtype=int)
            enodes    = [ nodes[nodeIDs.index(enodesID)] for enodesID in enodesIDs ]
            ElementType = int(linelist[1])
            if ElementType == 2 :
                addElement(elems, elemID, enodes)
                
            PhysicalGroup = int(linelist[3])    
            if PhysicalGroup == 2 : # Left edge
                Cons = numpy.array([1, 0])
                for node in enodes:
                    node.set_constraint(Cons)
                    LSnodes.append( node.get_ID() )
            elif PhysicalGroup == 3 : # Right edge
                Force = numpy.array([10.0, 0.0])
                for node in enodes:
                    node.set_forcecons(Force)
                    RSnodes.append( node.get_ID() )
        
        LSnodes = list(set(LSnodes)) 
        # Remove duplicates, contains node IDs with an arbitrary order
        RSnodes = list(set(RSnodes))
        
        
        mesh = Mesh( nodes, elems, LSnodes, RSnodes )
        mesh.get_node(1).set_constraint(numpy.array([1 ,1]))
    except:
        fin.close()

    return mesh


\end{verbatim}

\section{classParameter.py}
\begin{verbatim}
## @package class_parameter
#  This file contains all material parameters

## Parameter object
class Parameter:

    ## Constructor
    # @param E =                   Young's Modulus             [Pa]
    # @param nu =                   Poisson Ratio               []
    def __init__ ( self, E,nu):
        self.E=E
        self.nu=nu

\end{verbatim}

\section{FEM.py}
\begin{verbatim}
## @package FEM
#  This module contains the functions used for the FEM.

import numpy
from MeshDat import *
from classParameter import *
import scipy.sparse.linalg

## Distribute the Force over the nodes on the righthand side
#
#  @param  mesh This is a MeshDat.Mesh()
#  @param  Force Force on the nodes on the righthand side
#  @return       Distributed force on nodes
def Distribute_Force(mesh,Force):
    
    nnodes = len(mesh.get_RSnodes())
    c = 0
    f = Force / (nnodes-1)
    dis_force = numpy.zeros((nnodes,3))
    
    for node in mesh.get_RSnodes():
        if c<2:
        
            dis_force[c] = numpy.array([node.get_index(),f/2,0])
            
        else:
            
            dis_force[c] = numpy.array([node.get_index(),f,0])
        
        c+=1
        
    return dis_force

## Limit F to f where the entries of K in KU=F are known
#
#  @param  mesh This is a MeshDat.Mesh()
#  @param  Force Distributed force on nodes
#  @return       Limited F
def getF(mesh,Force):
    nnodes=mesh.get_nr_of_nodes()
    ff = numpy.zeros( (nnodes,2) )
    for iforce in range(len(Force)):
        ff[(Force[iforce,0]),0] = Force[iforce,1]
        ff[(Force[iforce,0]),1] = Force[iforce,2]
    F =  numpy.zeros( (2*nnodes,1) )
    for element in mesh:
         fe = numpy.zeros( 2*len(element) )

         node_ID = numpy.zeros(len(element))
         count1 = 0
         for node in element:
             node_ID[count1] = node.get_index()
             count1 += 1
         count2 = 0
         xis, ws = element.get_integration_scheme( 'gauss', 1 )
         for xi, w in zip( xis, ws ):                    #get Ke for each element
             N = get_shapes(xi)
             N2d = numpy.zeros((2,2*len(element)))
             for inode in range(0,len(element)):
                 N2d[0,0+2*inode] = N[inode]
                 N2d[1,1+2*inode] = N[inode]
             fe += w * ff[(node_ID[count2],)].dot(N2d)
             count2 += 1
         count3 = 0
         for node in element:

            F[2*node.get_index()] += fe[2*count3]
            F[2*node.get_index()+1] += fe[2*count3+1]
            count3 += 1
    for x in range(len(ff)):
        F[2*x]=ff[x,0]
        F[2*x+1]=ff[x,1]
    return F

## Rewrite K to Kn (rows with known U are removed) and solve the linear system
#
#  @param  mesh This is a MeshDat.Mesh()
#  @param  F Limited Force Distribution
#  @param  K Stiffness Matrix
#  @return Displacement U
def solveSys(mesh,F,K):
    nr_of_nodes_w_cons = mesh.get_nr_of_nodes_with_constraints()
    cons = numpy.zeros((nr_of_nodes_w_cons,3))
    consDis = numpy.zeros((nr_of_nodes_w_cons,3))
    nodes = mesh.get_nodes()
    c=0

    #read constraints
    for node in nodes:
        if node.get_constraint().sum()>0:
            cons[c,0:]=[node.get_index(), node.get_constraint()[0],
            	node.get_constraint()[1]]
            consDis[c,0:]=[node.get_index(), 0,0]
            c+=1
    #limit F to f. The values at f where the position of u is prescribed or 
    	fixed are pulled out
    nnodes = mesh.get_nr_of_nodes()
    boundryLen=numpy.sum(cons[:,1:])

    Kn=numpy.zeros((2*nnodes-boundryLen,2*nnodes-boundryLen))
    f=numpy.zeros(2*nnodes-boundryLen)
    ci=0
    for c in range(2*nnodes):
        row = numpy.where(cons[:,0] == c//2)[0]
        if row.size == 0 or cons[row,c.__mod__(2)+1] == 0:
            f[ci]=F[c]
            ci+=1

    # remove the rows and columns where u is known (prescribed), 
    	put them in the righthand side
    ci=0
    for c in range(2*nnodes):
        row = numpy.where(cons[:,0] == c//2)[0]
        if row.size == 0 or cons[row,c.__mod__(2)+1] == 0:
            cci=0
            for cc in range(2*nnodes):
                row2= numpy.where(cons[:,0] == cc//2)[0]
                if row2.size == 0 or cons[row2,cc.__mod__(2)+1] == 0:
                    Kn[ci,cci]=K[c,cc]
                    cci+=1
            ci+=1
        else:
            cci=0
            for cc in range (2*nnodes):
                row2= numpy.where(cons[:,0] == cc//2)[0]
                if row2.size == 0 or cons[row2,cc.__mod__(2)+1] == 0:
                    f[cci]-=K[cc,c]*consDis[row,c.__mod__(2)+1]
                    cci+=1

    #solve and replace the known displacements
    u=numpy.zeros(2*nnodes)

    sKn = scipy.sparse.csr_matrix(Kn)    
    
    sx = scipy.sparse.linalg.spsolve(sKn,f)

    # restore the full U, so add the prescribed displacements.
    ci=0
    for c in range(2*nnodes):
        row = numpy.where(cons[:,0] == c//2)[0]
        if row.size == 0 or cons[row,c.__mod__(2)+1] == 0:
            u[c]=sx[c-ci]
        if cons[row,c.__mod__(2)+1] == 1:
            u[c]=consDis[row,c.__mod__(2)+1]
            ci+=1
    return u


## Make stiffness matrix K from all separate Ke from the elements
#
#  @param  mesh This is a MeshDat.Mesh()
#  @param  param This is a class_parameter.Parameter()
#  @return Stifness K
def getK(mesh,param):
    C = param.E/((1-param.nu**2))*numpy.array([[1, param.nu, 0],
                                     [param.nu, 1, 0],
                                     [0, 0, (1-param.nu)/2]])
    nnodes=mesh.get_nr_of_nodes()
    K = numpy.zeros( (2*nnodes,)*2 )
    for element in mesh:
        Ke = numpy.zeros( (2*len(element),)*2 )
        xi, w = element.get_integration_scheme( 'gauss', 1 )

        #create Ke with the weight functions

        GradN = element.get_shapes_gradient( xi )

        GradN2d = numpy.zeros((3,2*len(element)))
        for inode in range(0,len(element)):
            GradN2d[0,0+2*inode] = GradN[inode,0]
            GradN2d[1,1+2*inode] = GradN[inode,1]
            GradN2d[2,0+2*inode] = GradN[inode,1]
            GradN2d[2,1+2*inode] = GradN[inode,0]
        Ke =  w * numpy.dot( numpy.dot( GradN2d.T, C ),GradN2d)
        counti = 0
        ## put the values of Ke in K at the node positions
        for nodei in element:
            countj = 0
            for nodej in element:
                for i1 in [0,1]:
                    for i2 in [0,1]:
                        K[2*nodei.get_index() +i1,2*nodej.get_index()+i2] 
                        	+= Ke[2*counti+i1,2*countj+i2]
                countj += 1
            counti += 1
    #symmetry check
    if ((K.round(3) != K.round(3).T).all()):
        print 'K is unsymmetric'
    return K



## returns a array of stresses in all elements in [xx yy xy]
#
#  @param  mesh This is a MeshDat.Mesh()
#  @param  U Displacement of nodes
#  @param  param This is a class_parameter.Parameter()
#  @return Stiffness K
def get_FEM_stresses(mesh,U,param):
    C = param.E/((1-param.nu**2))*numpy.array([[1, param.nu, 0],
                                     [param.nu, 1, 0],
                                     [0, 0, (1-param.nu)/2]])
    siglist = numpy.zeros((mesh.get_nr_of_elements(),3))
    elec=0
    for element in mesh:
        xi, w = element.get_integration_scheme( 'gauss', 1 )
        sig = 0
        q=numpy.zeros((6,1))
        qi=0
        for node in element:
            id=node.get_index()
            q[2*qi]=U[2*id]
            q[2*qi+1]=U[2*id+1]
            qi+=1

        GradN = element.get_shapes_gradient( xi )
        GradN2d = numpy.zeros((3,2*len(element)))
        for inode in range(0,len(element)):
            GradN2d[0,0+2*inode] = GradN[inode,0]
            GradN2d[1,1+2*inode] = GradN[inode,1]
            GradN2d[2,0+2*inode] = GradN[inode,1]
            GradN2d[2,1+2*inode] = GradN[inode,0]
        sig = w * numpy.dot( numpy.dot( C , GradN2d),q)
        siglist[elec,:]=sig.T
        elec+=1
    return siglist

def get_shapes ( xi ):
    return numpy.array([1-xi[0]-xi[1],xi[0],xi[1]])


\end{verbatim}

\section{output.py}
\begin{verbatim}
## @package output
#  This module contains the functions used for creating the output. 
The ouput is formatted in .VTK and can be opened with Paraview.

import numpy
import os
## plot_solution.
#  Plots the nodes, elements, stresses and displacements of the mesh
#  @param mesh This is a MeshDat.Mesh()
#  @param outfile Name of the outputfile
#  @param outfolder Location of the outputfile
#  @param U In U the displacements of the nodes are stated
#  @param sig In sig the stresses for all nodes are listed

def plot_solution( mesh, outfile, outfolder,U,sig):

    create_folder(outfolder)

    # Opening Tags
    output = open(outfolder+'/post_proc_'+str(outfile).zfill(3)+'.vtk','w+')
    output.write('# vtk DataFile Version 2.0 \nDiscrete Dislocations \nASCII \n\n')

    # Plot Nodes
    nrNodes=mesh.get_nr_of_nodes()
    nodes_list=mesh.get_nodes()
    output.write('DATASET UNSTRUCTURED_GRID \nPOINTS '+str(nrNodes)+' float\n')
    for c in range(nrNodes):
        output.write(' '.join(map(str,nodes_list[c].get_coordinate()+[U[2*c],
        	U[2*c+1]])))
        output.write(' 0.0\n')



    # Plot Elements
    nrElements=mesh.get_nr_of_elements()
    output.write('\nCELLS '+str(nrElements)+' '+str(nrElements*4)+'\n')
    for ele in mesh:
        elenodes=[]
        for node in ele:
            elenodes.append(node.get_index())
        output.write('3 ')
        output.write(' '.join(map(str,elenodes)))
        output.write('\n')

    # output Celltypes (every cell is a triangular element)
    output.write('\nCELL_TYPES '+str(nrElements)+'\n')
    for ele in mesh:
        output.write('5\n')


    # Plot stresses in cells
    output.write('\nCELL_DATA '+str(nrElements)+' \nSCALARS x-stress 
    	FLOAT \nLOOKUP_TABLE default\n')
    for c in range(nrElements):
        #output.write(str(nodes_list[c].get_stress_n()))
        output.write(str(sig[c,0]))
        output.write('\n')

    output.write(' \nSCALARS y-stress FLOAT \nLOOKUP_TABLE default\n')
    for c in range(nrElements):
        #output.write(str(nodes_list[c].get_stress_n()))
        output.write(str(sig[c,1]))
        output.write('\n')

    output.write(' \nSCALARS Mises-stress FLOAT \nLOOKUP_TABLE default\n')
    for c in range(nrElements):
        #output.write(str(nodes_list[c].get_stress_n()))
        output.write(vonMises(sig[c,:]))
        output.write('\n')

    #plot displacements in nodes
    output.write('\nPOINT_DATA '+str(nrNodes)+' \nSCALARS x-displacement FLOAT 
    	\nLOOKUP_TABLE default\n')

    for c in range(nrNodes):
        #output.write(str(nodes_list[c].get_stress_n()))
        output.write(str(U[2*c]))
        output.write('\n')

    output.write(' \nSCALARS y-displacement FLOAT \nLOOKUP_TABLE default\n')

    for c in range(nrNodes):
        #output.write(str(nodes_list[c].get_stress_n()))
        output.write(str(U[2*c+1]))
        output.write('\n')

    output.close()
    print 'File written to '+outfolder+'/post_proc_'+str(outfile).zfill(3)+'.vtk'

## create_folder
#  Creates output Folder if it does not exist already
# @param path defines the output path.

def create_folder(path):
    try:
        os.makedirs(path)
    except OSError:
        if not os.path.isdir(path):
            raise

## vonMises
#  Calculates the Von Mises stress on elements
#  @param st Vector containing the stress elements

def vonMises(st):
    stress = (1/0.70710678118)*numpy.sqrt((st[0]-st[1])**2+(st[1])**2+(st[0])
    	**2+6*st[2])
    return str(stress)
\end{verbatim}


\chapter{Gmsh}
\section{Geometry}
\begin{verbatim}
// Inputs
boxdim = 10;
gridsize = boxdim/30;
Z = 0; 

// Points
Point(1) = {0,0,Z,gridsize};
Point(2) = {boxdim,0,Z,gridsize};
Point(3) = {boxdim,boxdim,Z,gridsize};
Point(4) = {0,boxdim,Z,gridsize};

// Circles
C1x = 0.3*boxdim;
C1y = 0.3*boxdim;
R1 = 0.1*boxdim;
Point(5) = {C1x, C1y, 0, gridsize};
Point(6) = {C1x + R1, C1y, 0, gridsize};
Point(7) = {C1x - R1, C1y, 0, gridsize};

C2x = 0.3*boxdim;
C2y = 0.7*boxdim;
R2 = R1;
Point(8) = {C2x, C2y, 0, gridsize};
Point(9) = {C2x + R2, C2y, 0, gridsize};
Point(10) = {C2x - R2, C2y, 0, gridsize};

C3x = 0.7*boxdim;
C3y = 0.3*boxdim;
R3 = R1;
Point(11) = {C3x, C3y, 0, gridsize};
Point(12) = {C3x + R3, C3y, 0, gridsize};
Point(13) = {C3x - R3, C3y, 0, gridsize};

C4x = 0.7*boxdim;
C4y = 0.7*boxdim;
R4 = R1;
Point(14) = {C4x, C4y, 0, gridsize};
Point(15) = {C4x + R4, C4y, 0, gridsize};
Point(16) = {C4x - R4, C4y, 0, gridsize};

// Lines
Line(1) = {1, 2};
Line(2) = {2, 3};
Line(3) = {3, 4};
Line(4) = {4, 1};

Circle(5) = {6, 5, 7};
Circle(6) = {7, 5, 6};

Circle(7) = {9, 8, 10};
Circle(8) = {10, 8, 9};

Circle(9) = {13, 11, 12};
Line(10) = {12, 15};
Circle(11) = {15, 14, 16};
Line(12) = {16, 13};

// Line Loop
Line Loop(1) = {3, 4, 1, 2};
Line Loop(2) = {5, 6};
Line Loop(3) = {7, 8};
Line Loop(4) = {9, 10, 11, 12}; 

// Surface
Plane Surface(1) = {1};
Plane Surface(2) = {4,3,2,1};

// Create physical entities
Physical Surface("Surface") 	= {2};
Physical Line("Left Side") 		= {4};
Physical Line("Right Side") 	= {2};
Physical Line("Top and Bottom") = {3, 1};
Physical Line("Circle1")		= {5, 6};
Physical Line("Circle2")		= {7, 8};
Physical Line("Hole")			= {9, 10, 11, 12}; 

\end{verbatim}

\end{appendices}